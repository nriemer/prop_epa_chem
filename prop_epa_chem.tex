\documentclass[11pt]{article}
\usepackage{amsmath,amssymb,graphicx}
\usepackage[authoryear,sort&compress]{natbib}
\usepackage{calc}
\usepackage{float}
\usepackage{algorithmic}
\usepackage{hyperref}
\usepackage{pdfpages}
\usepackage{subfig}
\usepackage{overpic}
\usepackage{ragged2e}
\usepackage{array}
\usepackage{chemformula}
%\usepackage{showkeys}
\usepackage[margin=1in]{geometry}

%%%%%%%%%%%%%%%%%%%%%%%%%%%%%%%%%%%%%%%%%%%%%%%%%%%%%%%%%%%%%%%%%%%%%%%%%%%%%%

\hypersetup{colorlinks=true, linkcolor=black,  anchorcolor=black,
citecolor=black, filecolor=black, menucolor=black, pagecolor=black,
urlcolor=black}

%%%%%%%%%%%%%%%%%%%%%%%%%%%%%%%%%%%%%%%%%%%%%%%%%%%%%%%%%%%%%%%%%%%%%%%%%%%%%%

\newbox\qqboxa
\def\vcent#1{\setbox\qqboxa\hbox{#1}\raise -0.5\ht\qqboxa\hbox{#1}}

%%%%%%%%%%%%%%%%%%%%%%%%%%%%%%%%%%%%%%%%%%%%%%%%%%%%%%%%%%%%%%%%%%%%%%%%%%%%%%
% todos

\newcommand{\todo}[1]{\vspace{5 mm}\par \noindent
\marginpar{\textsc{Todo}}
\framebox{\begin{minipage}[c]{0.95 \textwidth}
\tt \flushleft #1 \end{minipage}}\vspace{5 mm}\par}

\newcounter{edremcounter}
\setlength\marginparwidth{18mm}
\newcommand{\smalltodo}[1]{%
  \addtocounter{edremcounter}{1}%
    \mbox{${}^{{\sf\theedremcounter}}$}%
  \marginpar{\vskip 3mm\footnotesize\flushleft
      \textsf{\textbf{\mbox{\theedremcounter
            }}}~{\fontsize{6}{10pt}\textsf{#1}}}}

%%%%%%%%%%%%%%%%%%%%%%%%%%%%%%%%%%%%%%%%%%%%%%%%%%%%%%%%%%%%%%%%%%%%%%%%%%%%%%
% captions

\makeatletter

\long\def\@makecaption#1#2{%
  \vskip\abovecaptionskip
  \small
  \centerline{\parbox[t]{0.85\textwidth}{#1: #2}}\par
  \vskip\belowcaptionskip}

\makeatother

%%%%%%%%%%%%%%%%%%%%%%%%%%%%%%%%%%%%%%%%%%%%%%%%%%%%%%%%%%%%%%%%%%%%%%%%%%%%%%
% DOIs in the references

\newcommand{\doi}[1]{DOI: \href{http://dx.doi.org/#1}{#1}}
\newcommand{\srefid}[1]{SRef-ID: \href{http://direct.sref.org/#1}{#1}}

%%%%%%%%%%%%%%%%%%%%%%%%%%%%%%%%%%%%%%%%%%%%%%%%%%%%%%%%%%%%%%%%%%%%%%%%%%%%%%

\graphicspath{{./graphics/}}

\begin{document}

\section{Overview of Science Questions, Objectives, and Tasks}
%1-1.5 pages to tell the whole story.

\section{Background}

%[3 pages] 


\section{Preliminary Work}
\label{sec:tools}

%[3 pages] What we have done so far:
%PartMC, WRF-PartMC

\subsection{WRF-PartMC: Particle-resolved modeling on the regional scale}

\paragraph{The particle-resolved aerosol model PartMC-MOSAIC.}
PartMC-MOSAIC is a stochastic, particle-resolved aerosol model
\citep{Riemer2009a,Zaveri2008}, which most recently has been coupled
with WRF to yield the first 3D particle-resolved model on the regional
scale (WRF-PartMC, \citep{Curtis2017,Curtis2019}).
%While the
%box model version is a convenient tool for exploring sensitivities on
%the process-level, it is limited by not resolving the spatial
%structure of the atmosphere. For this project we will therefore make
%use of the box model (0-D) capability as well as of WRF-PartMC on the
%regional scale.
The PartMC model provides the framework for our particle-resolved
aerosol representation and resolves the composition of many individual
particles within a well-mixed air parcel. Over the course of the
simulation, the mass of each constituent species within each particle
is tracked. Emission, nucleation and Brownian coagulation are
simulated with a stochastic Monte Carlo approach. While we predict the
chemical composition of individual particles, we currently do not have
the capability to predict {\color{red}{particle morphology}}, which is the
  focus of the proposed work.

PartMC is coupled with the state-of-the-art aerosol chemistry model
MOSAIC (Model for Simulating Aerosol Interactions and Chemistry,
\citep{Zaveri2008}) which includes the following components: the gas
phase photochemical mechanism CBM-Z \citep{Zaveri1999}, the
Multicomponent Taylor Expansion Method (MTEM) for estimating activity
coefficients of electrolytes and ions in aqueous solutions
\citep{Zaveri2005a}, the multi-component equilibrium solver for
aerosols (MESA) for intraparticle solid–liquid partitioning
\citep{Zaveri2005b}, the adaptive step time-split Euler method (ASTEM)
for dynamic gas-particle partitioning \citep{Zaveri2008}, a fully
dynamic gas-particle mass transfer treatment for SOA (as opposed to
bulk equilibrium) based on the SORGAM scheme \citep{Schell2001}. The
CBM-Z gas phase mechanism treats a total of 77 gas species. MOSAIC
treats key aerosol species including sulfate, nitrate, ammonium,
chloride, carbonate, methanesulfonic acid, sodium, calcium, ``other
inorganic mass'' (this represents species such as $\rm SiO_2$, metal
oxides and other unmeasured or unknown inorganic species present),
black carbon, primary organic carbon and secondary organic aerosol
(SOA). SOA includes reaction products of aromatic precursors, higher
alkenes, $\alpha$-pinene and limonene.

We have used the PartMC-MOSAIC box model extensively for process
studies of aerosol aging in different environments
\citep{Fierce2013,Fierce2015,Riemer2010,Tian2014}, and as a
benchmarking tool to quantify error in more approximate aerosol models
\citep{ching2016,Kaiser2014,McGraw2008}, frequently using large
scenario libraries \citep{Fierce2016,Hughes2018}.

\paragraph{Particle-resolved modeling on the regional scale.} WRF-PartMC
simulates individual aerosol particles within each grid cell of a
3-dimensional modeling domain (see Figure~\ref{fig:wrf_partmc} for a
simulation that covers the model domain of California), where we use the WRF
model to obtain the meteorological fields. The transport between grid cells by
mean-wind advection and turbulent diffusion is simulated by sampling the
particle population with appropriate probabilities.
Figure~\ref{fig:wrf_partmc} illustrates the level of detail that can be
achieved with WRF-PartMC. For each grid cell in the model domain, we store the
full aerosol state using approx. 10,000 computational particles, resulting in
7.5 billion particles in total. This allows us to retrieve the full spatial
distribution of aerosol mixing state, down to the composition of individual
particles and their source contributions. From this information, aerosol
properties such as CCN concentration or optical properties can be calculated,
building up from the per-particle level. Figure~\ref{fig:wrf_partmc}a shows
that high number concentrations are present near major highways since traffic
is a major particle source. While aerosol number concentration is a
fundamental bulk quantity, common to any chemical transport model, the
particle-resolved aerosol representation provides unprecedented detail for
particle composition and source tracking. Figure~\ref{fig:wrf_partmc}b shows
an example of the complex continuum of aerosol composition that exists within
the single grid cell marked with a star in Figure~\ref{fig:wrf_partmc}a;
particles of similar diameters can have very different chemical composition---
information that is usually lost when using traditional aerosol models.
WRF-PartMC simulations require a petascale resource with tens of thousands of
cores, fast interconnections between those cores, and sufficient memory per
core. We are currently using allocations on Blue Waters [operated by National
Center for Supercomputing Applications (NCSA) at the University of Illinois
(UIUC)] for these computationally demanding simulations. As Blue Waters is
going to be phased out in the near future, we will seek allocations on the
next-generation systems such as Theta (Argonne National Laboratory) and
Stampede2 (Texas Advanced Computing Center). See ``Facilities'' for more
details.

\paragraph{Integrated Abstract Chemical Mechanism}
Decades of progress in the identification of increasingly complex
atmospherically relevent mixed-phase physiochemical processes have resulted
in an advanced understanding of the evolution of atmospheric systems, but
have also introduced a level of complexity that few atmmospheric models were
originally designed to handle. In practical terms, this means that most
regional and global models comprise a collection of chemistry and
`chemistry-adjacent' software modules (e.g., those for gas-phase chemistry,
gas--aerosol partitioning, surface chemistry, condensed-phase chemistry,
partitioning between condensed phases, cloud droplet formation, cloud
chemistry, emissions, deposition, and photolysis)
along with `support' modules that calculate parameters needed by these
modules (e.g., vapor pressures, Henry's law constants, and activity
coefficients).
These software modules have, in most cases, been developed independently and
with a focus on computational efficiency that often leads to significant
development efforts when modules are coupled for the first time,
or changes to the underlying mechanisms are applied.
Although efforts have been made to standardize Earth
science module integration \citep{jockel2005}, these typically
retain the stand-alone nature of individual modules.
In addition, any module that treats processes
involving aerosols is typically tightly tied to the aerosol representation
of the host model (bins, modes, etc.). This makes porting comprehensive
physiochemical mechanisms between models with different aerosol
representations difficult.

Advances in computing hardware and software design coupled with the
complexity of multi-phase atmospheric processes make a rethinking of
the treatment of physiochemical processes in atmospheric models timely.
We have developed a prototype software package for the treatment of
physical and chemical processes of gases and aerosols built on the science
of existing atmospheric models but designed for
{\bf scalability} of the physiochemical mechanisms, {\bf portability} across
a wide variety of atmospheric models, and {\bf efficiency} of runtime
calculations. This is accomplished through a fully object-oriented approach
to the design of abstract physiochemical systems and mechansims, the use of
\texttt{json} files for runtime model configuration, and a structure
designed for the use of JIT compiling and deployment to accelorator
computing platforms.

\begin{description}
\item [Object-oriented design.] At the core of the chemistry package is an
abstract chemical mechanism made up of instances of classes that extend one
of three abstract base classses: \texttt{Process},
\texttt{Parameter}, and \texttt{AerosolRepresentation}.
A list of currently implemented and
planned extending classes is shown in Table~\ref{tab:model_classes}.
\texttt{Process} extending classes describe physical and chemical
transformations (reactions, phase transfer, etc.) and provide functions
required during solving to advance the model state in time. \texttt{Parameter}
extending classes provide parameters needed by processes during solving
(e.g., vapor pressures, Henry's law constants, activity coefficients).

\begin{table}[]
\caption{Current and planned model classes in the integrated abstract chemical
         mechanism.}
\label{tab:model_classes}
\begin{tabular}{lll}
 \hline
 Base class & Description                                      & References \\
 \hline
 Process    & Arhennius reaction                               & --- \\
 Process    & Troe reaction                                    & --- \\
 Process    & Henry's Law phase transfer                       & --- \\
 Process    & Vapor-pressure based phase transfer              & SIMPOL.1 \citep{Pankow2008} \\
 Process    & Photolysis                                       & external \\
 Process    & Emissions                                        & external \\
 Process    & Deposition                                       & --- \\
 Process    & Condensed-phase reversible reaction              & --- \\
 Process    & Condensed-phase forward reaction                 & --- \\
 Process    & Surface reaction\textsuperscript{*}              & --- \\
 Parameter  & PD-FiTE activity coefficients                    & \cite{Topping2009} \\
 Parameter  & UNIFAC activity coefficients                     & \cite{Hansen1991} \\
 Parameter  & AIOMFAC activity coefficients\textsuperscript{*} & \cite{Zuend2008} \\
 AerosolRepresentation & Binned/Modal                          & --- \\
 AerosolRepresentation & Particle resolved                     & PartMC \citep{Riemer2009a} \\
 \hline
 \textsuperscript{*}plannned
\end{tabular}
\end{table}

\item [Run-time configuration.]

\item [Condensed memory structure.]

\end{description}


\begin{figure}
  \centering
  \includegraphics[scale=0.5]{wrf_partmc_results_new}
  \caption{(a) Snapshot of the spatial distribution of simulated
      number concentration for a WRF-PartMC run. (b) For each grid
      cell, we store the full aerosol state using 5,000 computational
      particles. Here we show the two-dimensional number distribution
      as a function of dry diameter and organic aerosol mass fraction
      for the location indicated with a star in (a). (c) Composition
      of a particle sampled from the population shown in (b). (d) This
      particle has coagulated xx times, with the individual components
      coming from the source sectors as shown in the pie
      chart. \label{fig:wrf_partmc}}
\end{figure}

\section{Proposed Research and Tasks}
\label{sec:prop-rese-tasks}
%[4 pages]

\subsection{Task 1: Creating the computational framework for benchmark transport and microphysics}
This is where Matt's figure goes plus the description of how to handle
the various phases and what he has done for MONARCH. Adding
per-particle transport could go into here as well. This avoids
numerical diffusion, making this a very benchmarky benchmark model.

\begin{description}
\item[Task 1a:] Implement per-particle transport
\item[Task 1b:]
\item[Task 1c:]
\end{description}

\subsection{Task 2: Implementing dynamic treatment of inorganic/organic mixtures}
This is fiddling with the MOSAIC part.
\begin{description}
\item[Task 2a:]
\item[Task 2b:]
\end{description}

\subsection{Task 3: Application of WRF-PartMC$^+$ to CARES and SOAS domains}
Run and do error quantification
\begin{description}
\item[Task 3a:]
\item[Task 3b:]
\end{description}

%Our vision is captured in Figure~\ref{fig:prl}. This is inspired by the
%\begin{figure}
%  \centering
%  \includegraphics[scale=0.5]{prl}
%  \caption{adfasdfasdf \label{fig:prl}}
%\end{figure}
%This proposal focuses on getting from PRL3 to PRL4 and 5.


\section{Timetable of Activities and Project Management}
[0.5 pages]

\bibliographystyle{plain-local-srefid} \bibliography{refs}

\end{document}


